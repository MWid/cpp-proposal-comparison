\documentclass[a4paper,11pt,final]{article}

\usepackage[american]{babel} % needed for iso dates
\usepackage[iso,american]{isodate} % use iso format for dates
\usepackage[final]{listings} % code listings
\usepackage{longtable} % auto-breaking tables
\usepackage{booktabs} % fancy tables
\usepackage{relsize} % provide relative font size changes
\usepackage{underscore} % remove special status of '_' in ordinary text
%\usepackage{verbatim} % improved verbatim environment
%\usepackage{alltt}
\usepackage{parskip} % handle non-indented paragraphs "properly"
\usepackage{array} % new column definitions for tables
\usepackage[normalem]{ulem}
\usepackage[svgnames,table,hyperref]{xcolor} % define colors for strikeouts and underlines
\usepackage{amsmath} % additional math symbols
\usepackage{mathrsfs} % mathscr font
\usepackage{multicol}
\usepackage{fixltx2e}
\usepackage{xspace}
\usepackage{fixme}
\usepackage[final]{graphicx}
\usepackage[unicode=true,
pdftitle={Explicitly defaulted relational and equality operators},
pdfsubject={Proposal — Programming Language C++, Language Evolution Working Group},
pdfauthor={Marcel Wid},
bookmarks=true,
bookmarksnumbered=true,
pdfpagelabels=true,
pdfpagemode=UseOutlines,
pdfstartview=FitH,
linktocpage=true,
colorlinks=true,
linkcolor=Red,
citecolor=Navy,
urlcolor=Blue,
plainpages=false,
pdfpagelabels=true,
]{hyperref}

%\usepackage{lmodern}
%\usepackage[T1]{fontenc}

\usepackage{fontspec}
\pdfprotrudechars=2
\pdfadjustspacing=2
\usepackage{luatextra}
\usepackage{microtype}
\defaultfontfeatures{Ligatures=TeX}
\setmainfont[Numbers=OldStyle]{Cambria}
\setsansfont[Scale=MatchLowercase,Numbers=OldStyle]{Calibri}
\setmonofont[Scale=MatchLowercase]{Consolas}
\usepackage[math-style=TeX]{unicode-math}
\setmathfont{Cambria Math}

\usepackage{tocloft}
\usepackage{titling}
\usepackage{titlesec}
\usepackage[pdftex, top=2cm, bottom=2cm, left=2cm, right=2cm]{geometry}
\usepackage[absolute]{textpos}
\setlength{\TPHorizModule}{1cm}
\setlength{\TPVertModule}{1cm}

\usepackage{natbib}

\renewcommand{\thesection}{\Roman{section}.}
\cftsetindents{section}{0em}{2em}

%%--------------------------------------------------
%% Environments for code listings.

% We use the 'listings' package, with some small customizations.  The
% most interesting customization: all TeX commands are available
% within comments.  Comments are set in italics, keywords and strings
% don't get special treatment.

\definecolor{directiveclr}{RGB}{128,64,0}
\definecolor{keywordclr}{RGB}{0,0,255}
\definecolor{identifierclr}{RGB}{0,0,0}
\definecolor{emphstyleclr}{RGB}{0,128,192}
\definecolor{constantclr}{RGB}{164,0,164}
\definecolor{stringstyleclr}{RGB}{220,20,60} %{128,128,128}
\definecolor{commentclr}{RGB}{0,128,0}

\lstset{language=C++,
        basicstyle=\footnotesize\ttfamily,
				backgroundcolor=\color{gray!20},
        keywordstyle=\bfseries\color{keywordclr},
				directivestyle=\color{directiveclr},
				commentstyle=\itshape\color{commentclr},
				identifierstyle=\color{identifierclr},
				emph={Author, Book, string, vector},
				emphstyle=\color{emphstyleclr},
				emph=[2]{CONSTANT},
				emphstyle=[2]\color{constantclr},
				stringstyle=\color{stringstyleclr},
				moredelim=[is][\color{directiveclr}]{_}{_},
        xleftmargin=7em,
				xrightmargin=7em,
        showstringspaces=false,
        columns=flexible,
        keepspaces=true,
        texcl=true}

% Code and definitions embedded in text.
\newcommand{\tcode}[1]{\lstinline{#1}}

\newcommand{\Rplus}{\protect\hspace{-.1em}\protect\raisebox{.1ex}{\small+}}
\newcommand{\Cpp}{\mbox{C\Rplus\Rplus}\xspace}

\setlength{\droptitle}{7em}
\title{Explicitly defaulted relational and equality operators}
\author{}
\date{}

\begin{document}

\begin{textblock}{12}(8.5,2)
\begin{tabular}{l l}
\textbf{Document number:} & NXXXX\\
\textbf{Date:} & \today\\
%\textbf{Revises:} & N4109\\
\textbf{Project:} & Programming Language \Cpp,\\
& Language Evolution Working Group\\
\textbf{Reply to:} & Marcel Wid\\
& <\href{mailto:marcel.wid@ods-solutions.de}{marcel.wid@ods-solutions.de}>
\end{tabular}
\end{textblock}

\maketitle

\tableofcontents

\section{Introduction}
Sorting and searching is a common task in everyday programming. In order to be able to search or sort objects of a type \tcode{T}, the type \tcode{T} must provide relational and equality operators. For convenience of the users of type \tcode{T} it is necessary to implement all variants of these operators, which results in a lot of boilerplate code. This is needlessly verbose, error prone and contrary to modern \Cpp. Generally it is enough to provide only one relational operator, namely \tcode{operator<=()}, and all other relational and equality operators can be defined in terms of \tcode{operator<=()}. We propose an extension of the core language by allowing the relational and equality operators to be explicitly defaulted.

\section{Motivation and Scope}
Consider the following snipplet:

\begin{lstlisting}
#include _<string>_
#include _<vector>_

struct Author
{
    std::string lastname;
    std::string firstname;
};

struct Book
{
    std::string title;
    Author author;
    std::string publisher;
    unsigned short year;
};

std::vector<Book> books;
\end{lstlisting}

How do we want to sort the vector \tcode{books}? One reasonable possibility is to use 


\section{Design Decisions}

\section{Relation to N4126}
\cite{oleg3}\cite{oleg1}\cite{oleg2}
\begin{lstlisting}
struct A
{
  short x;
  int   y;
  char  z;
};

struct B
{
  char  z;
  short x;
  int   y; 
};

struct C
{
  int   y;
  short x;
  char  z;
};

int main()
{
  std::cout << sizeof(A) << "\n" << sizeof(B) << "\n" << sizeof(C);
}
\end{lstlisting}

\section{Technical Specifications}

\section{Acknowledgments}

\addcontentsline{toc}{section}{References}
\bibliographystyle{newapa}
\bibliography{references}

\end{document}
